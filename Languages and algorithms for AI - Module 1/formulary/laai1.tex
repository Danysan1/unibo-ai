\documentclass[]{article}
\usepackage[a4paper, portrait, margin=0.75in]{geometry}
\usepackage{amsmath}
\usepackage{amssymb}
\usepackage{graphicx}
\usepackage{tabularx}
\setlength\parindent{0pt}

%opening
\title{Formulary \\ \large Languages and Algorithms for Artificial Intelligence - Module 1}
\author{Daniele Santini}

\begin{document}

%\maketitle

Symbolic AI systems are based on logic reasoning; they can provide explanations for decisions. Sub-symbolic AI systems are based on machine learning and artificial neural networks; they cannot provide explanations for decisions.

\section{Propositional logic}

\subsection{Semantics}

\noindent $\Gamma = F_1,\dots,F_n$ (\textit{axioms}/\textit{promises});
$F$ \textit{conclusion}; 	
$\Gamma \models F \iff F$ is a \textbf{logical consequence} of $\Gamma$.

\noindent \textbf{Valid} formula $\iff$ \textbf{Tautology} $\iff$ True in any circumstance.

\noindent \textbf{Inconsistent} formula $\iff$ \textbf{Unsatisfiable} formula $\iff$ False in any circumstance.

\noindent $G$ valid $\iff \neg G$ inconsistent.

\begin{table}[h]
	\centering
	\caption{Truth table of main logical connectives}
	\begin{tabular}{ |c |c|c | c | c| c | c | } 
		\hline
		 &  & \textbf{Not} & \textbf{And} & \textbf{Or} & \textbf{Implication} & \textbf{Double implication} \\
		\hline
		$P_1$ & $P_2$ & $\neg P_1$ & $P_1 \land P_2$ & $P_1 \lor P_2$ & $P_1 \rightarrow P_2$ & $P_1 \leftrightarrow P_2$ \\
		\hline
		T & T & F & T & T & T & T \\
		\hline
		T & F & F & F & T & F & F \\
		\hline
		F & T & T & F & T & T & F \\
		\hline
		F & F & T & F & F & T & T \\
		\hline
	\end{tabular}
\end{table}

\noindent $F$ \textbf{logically equivalent} to $G \iff F \equiv G \iff (F \models G$ and $G \models F)$.

\subsection{Calculus}

\begin{table}[h]
	\centering
	\caption{Logical equivalence rules}
	\begin{tabular}{ | r  c  l  c | } 
		\hline
		$P \land Q$ & $\equiv$ & $Q \land P$ & Commutativity of AND\\
		\hline
		$P \lor Q$ & $\equiv$ & $Q \lor P$ & Commutativity of OR\\
		\hline
		$(P \land Q) \land R$ & $\equiv$ & $P \land (Q \land R)$ & Associativity of AND\\
		\hline
		$(P \lor Q) \lor R$ & $\equiv$ & $P \lor (Q \lor R)$ & Associativity of OR\\
		\hline
		$\neg(\neg P)$ & $\equiv$ & $P$ & Double-negation elimination\\
		\hline
		$P \rightarrow Q$ & $\equiv$ & $\neg P \rightarrow \neg Q$ & Contraposition\\
		\hline
		$P \rightarrow Q$ & $\equiv$ & $\neg P \lor Q$ & Implication elimination\\
		\hline
		$P \leftrightarrow Q$ & $\equiv$ & $(P \rightarrow Q) \land (Q \rightarrow P)$ & Biconditional elimination \\
		\hline
		$\neg(P \land Q)$ & $\equiv$ & $\neg P \lor \neg Q$ & De Morgan's law *\\
		\hline
		$\neg(P \lor Q)$ & $\equiv$ & $\neg P \land \neg Q$ & De Morgan's law *\\
		\hline
		$P \land (Q \lor R)$ & $\equiv$ & $(P \land Q) \lor (P \land R)$ & Distributivity of AND over OR\\
		\hline
		$P \lor (Q \land R)$ & $\equiv$ & $(P \lor Q) \land (P \lor R)$ & Distributivity of OR over AND\\
		\hline
	\end{tabular}

	* De Morgan's law can be generalized to any arbitrary finite number of connected variables.
\end{table}

\noindent \textbf{Conjunctive Normal Form}: $F = F_1 \land F_2 \land ... \land F_n $ where $F_i = F_{i1} \lor F_{i2} \lor ... \lor F_{ik} $ (disjunction of atoms).

\noindent \textbf{Disjunctive Normal Form}: $F = F_1 \lor F_2 \lor ... \lor F_n $ where $F_i = F_{i1} \land F_{i2} \land ... \land F_{ik} $ (conjunction of atoms).

\noindent \textbf{Deduction Theorem}: $(F_1 \land F_2 \land ... \land F_n) \models G \iff (\models(F_1 \land F_2 \land ... \land F_n)) \rightarrow G$

\noindent \textbf{Proof by refutation}: $(F_1 \land F_2 \land ... \land F_n) \models G \iff F_1 \land F_2 \land ... \land F_n \land \neg G $ inconsistent.

\subsubsection{Natural deduction}

\begin{table}[h]
	\centering
	\caption{Introduction and elimination rules for main logical connectives}
	% https://www.cs.cornell.edu/courses/cs3110/2012sp/lectures/lec15-logic-contd/lec15.html
	\begin{tabular}{ |c |c|c | } 
		\hline
		& \textbf{Introduction} & \textbf{Elimination} \\
		\hline
		\textbf{And} & $\dfrac{\varphi \hspace{1em} \theta}{\varphi \land \theta}\land I$ & $\dfrac{\varphi \land \theta}{\varphi}\land E \hspace{3em} \dfrac{\varphi \land \theta}{\varphi}\land E$ \\
		\hline
		\textbf{Or} & $\dfrac{\varphi}{\varphi \lor \theta} \lor I \hspace{3em} \dfrac{\theta}{\varphi \lor \theta} \lor I$ & $\dfrac{\varphi \lor \theta \hspace{1em} \varphi \rightarrow \psi \hspace{1em} \theta \rightarrow \psi}{\psi} \lor E$ \\
		\hline
		\textbf{Implication} & $\dfrac{\begin{array}{c} [\varphi] \\ \vdots \\ \theta \end{array}}{\varphi \rightarrow \theta} \rightarrow I$ & $\dfrac{\varphi \hspace{1em} \varphi \rightarrow \theta}{\theta} \rightarrow E$ (\textit{modus ponens}) \\
		\hline
	\end{tabular}
\end{table}

\noindent Ex falso sequitur quodlibet: $\dfrac{\bot}{\varphi}\bot$. Reductio ad absurdum: $\dfrac{\begin{array}{c} [\neg \varphi] \\ \vdots \\ \bot \end{array}}{\varphi} RAA$.

%\noindent $\begin{cases}
%	\Gamma \models \varphi \\
%	\Gamma = \emptyset
%\end{cases} \Rightarrow \models \varphi \Rightarrow \varphi$ theorem.

\noindent $\Gamma \models \varphi \iff \Gamma \models \varphi$ (Completeness theorem: $\Gamma \models \varphi \Rightarrow \Gamma \models \varphi$; Soundness theorem: $\Gamma \models \varphi \Leftarrow \Gamma \models \varphi$).

\subsubsection{Resolution}

\noindent Refutation theorem: $\theta \models \psi \iff \not\models \psi \land \neg\theta$

\noindent Resolution: $\dfrac{R \lor A \hspace{1em} R' \lor \neg A'}{R \lor R'}$

\section{First Order Logic}

TODO

\section{Logic Programming}

TODO

$ A \doteq B \Leftrightarrow A $ unifiable with $ B $

Let $A_1, \dots, A_k$ be atomic formulas; A unifier for the set ${A_1,\dots, A_k}$ is a substitution $\sigma$ such that $A_1 \sigma = A_2 \sigma = \dots = A_k \sigma$, where = represents the property of being the identical formula.
A unifier $\sigma$ is said to be a Most General Unifier (MGU) if, for every unifier $\tau$ for the same set, there is a unifier $\rho$ such that $\tau = \sigma \rho$ . From the MGU you can derive any other unifier.

\end{document}
